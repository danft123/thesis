% config padrao
\documentclass[notitlepage]{article} 
\usepackage[brazilian]{babel}
\usepackage[utf8]{inputenc}
\usepackage[T1]{fontenc}
\usepackage{tikz}
\usepackage{csquotes} %If using biblatex
\usepackage{tikz-cd}
\usepackage{dsfont}
\usetikzlibrary{babel}
\usetikzlibrary{matrix}
\usepackage{amsthm}
\usepackage{amsmath}
\usepackage{hyperref}
\usepackage{amssymb}
\usepackage{color}
\usepackage{mathtools}
\usepackage{array}
\usepackage{comment}
\usepackage[thinlines]{easytable}
\usepackage{indentfirst}
\usepackage{titling}
\usepackage{enumitem}
\usepackage{biblatex} %Imports biblatex package
% bibliografia
\addbibresource{references.bib} %Import the bibliography file
% ambientes customizados
\newtheorem{theorem}{Theorem}
\newtheorem{corollary}{Corollary}
\newtheorem{lemma}[theorem]{Lemma}
\newtheorem{definition}{Definition}
\newtheorem{proposition}{Proposition}
\newtheorem{remark}{Remark}
\newtheorem{example}{Example}
\newtheorem{dem}{Proof}
\newtheorem*{dem*}{Proof}
% comandos customizados
\newcommand{\ihom}{\mathcal{H}om}
\newcommand{\iext}{\mathcal{E}xt}
\newcommand{\PS}{{\mathbb{P}^2}}
\newcommand{\SO}{\mathcal{O}_{\mathbb{P}^2}}
\newcommand{\SF}{\mathcal{F}}
\newcommand{\SG}{\mathcal{G}}
\newcommand{\SE}{\mathcal{E}}
\newcommand{\deriv}{\partial_{x_i}}
\DeclarePairedDelimiter\ceil{\lceil}{\rceil}
\DeclarePairedDelimiter\floor{\lfloor}{\rfloor}
% cabeçalho
\title{Arranjos}
\author{danielrag}
\date{March 2020}
% inicio do documento
\begin{document}
% modulos
    %% intro
        %% syzygies and gradient
        \section{Syzygies and Gradients}
\subsection{Dual Curves and Polars}
\begin{definition}[Dual Projective Space]
Let $P^n$ be a projective space with homogeneous coordinates $(x_0;,,,;x_n)$, then the dual projective space ${P^n}^*$ is a projective space with coordinates $(\lambda_0;,,,;\lambda_n)$ such that $\lambda_i$ represents a hyperplane $\sum_i \lambda_i x_i = 0$.
\end{definition}
Given a homogeneous polynomial $f \in H^0 \SO(d)$ there is a rational map $\nabla f: \mathbb{P}^n \to (\mathbb{P}^n)^*$ taking $p \mapsto (\partial_i f(p))$.
\begin{definition}[Dual Variety]
Let $X = V(f)$ be a irreducible projective variety. We define the dual variety $X^*$ to be the image of $\nabla f:X_{sm} \to (\mathbb{P}^n)^*$ where $X_{sm} \subset X$ is the smooth locus.
\end{definition}
\begin{remark}
The dual variety $X^*$ can be see as a variety in $\mathbb{P}^n \times (\mathbb{P}^n)^*$ described as the closure of pairs $(p,H)$ such that $p \in X_{sm}$ and $H$ is the tangent hyperplane to $X$ at $p$.
\end{remark}
Let $C \subset \PS$ be a plane curve defined by a square-free polynomial $f$ and consider the rational map $\nabla f$, there is an unique square-free polynomial $f^*$ such that $V(f^*) = C^*$. Duality operation also commutes with factorization of $f$.
\begin{definition}
Consider $d^* = deg C^*$ called the class of $C$, using Plücker formula gives $d^*=d(d-1)-2\delta-3\kappa$ where $\delta$ is the number of ordinary nodes and $\kappa$ are ordinary cusps, higher order singularites are counted multiple times as ordinary singularities.
$d^*$ represents the number of tangent lines to $C$ passing through a general point in $\mathbb{P}^2 - C$
\end{definition}

\begin{remark}
Let $p \in C$ be non-singular, then the tangent of a curve at $p$ is given by equation ($f_i$ are the partial derivatives of $f$):
\begin{equation}\label{tangent_eq}
({f_0}_{|_p}) x + ({f_1}_{|_p}) y + ({f_2}_{|_p}) z = 0
\end{equation}
Let $a \in \PS$ and consider the following equation:
\begin{equation}\label{fixedpt_eq}
({f_0}) a_0 + ({f_1}) a_1 + ({f_2}) a_2 = 0
\end{equation}
The solutions for equation (\ref{fixedpt_eq}) intersected with $C$ are $p \in C$ such that $T_p C \in \PS^*$ passes through $a$, we will denote this set by $C_a$. It is called the polar to the curve $C$ at $a$.
\end{remark}

\begin{remark}\label{fixedpt_remark}
Using the same notation from our last remark, let $a \in \PS$ such that
\begin{equation}
({f_0}) a_0 + ({f_1}) a_1 + ({f_2}) a_2 = 0 \ \forall p \in C
\end{equation}
Then $C_a = C$, which implies that every tangent line must contain $a$
\end{remark}

\subsection{Syzygies}
\textcolor{red}{we might need to put this subsection after we talk about ideal sheaves, this is done in moduliconstruction}
\begin{remark}
Let $\nabla f: 3 \SO \to \SO(d)$ and $\mathcal{T}_C$ be the logarithmic sheaf defined as the kernel of $\nabla f$, it is a vector bundle of rank $2$. Let $a$ be a generator of degree $l$ of $Syz(\nabla f)$, hence $a = (a_0, a_1, a_2)$ with $a_i \in H^0 \SO (l)$ and:
\begin{equation}\label{syzygy_eq}
    \nabla f . a = 0
\end{equation}
If $l = 0$ then (\ref{syzygy_eq}) implies that $C_a = C$ (remark \ref{fixedpt_remark}), it follows that every tangent line must pass through $a$.
If $l>0$ then for every $p \in C$ we have
\begin{equation}
    \nabla f (p) . a(p) = 0
\end{equation}
In other words: $a(p) \in T_p C \subset \PS^*$
\end{remark}

    %% beginning
        %% moduli construction
        \section{Logarithmic Sheaves}
Our objective is to define a moduli space parameterizing logarithmic sheaves attached to hypersurfaces in $\mathbb{P}^2$. Suppose we are given a square-free polynomial $f \in H^0 \SO(d+1)$ and let $\nabla f: 3\SO \to \SO(d)$ be the Jacobian matrix, then $\nabla f$ defines a map $\phi:3\SO \to im(\nabla f)$ with $im(\nabla f) = \mathcal{I}_J(d)$ where $J$ is the variety defined by the partial derivatives of $f$.

\begin{remark}
If $J \not= \emptyset$ then it is a zero-dimensional scheme since $\mathcal{I}_J$ is torsion-free (the codimension of the singularity set of a coherent and torsion-free must be at least $2$).
\end{remark}


\subsection{Before Hilbert polynomial! merge sections after writing}
Fix an integer $d>0$, let $W_d = Hom(3\SO,\SO(d))$ as vector space and let $Quot(3\SO)$ be the Quot scheme representing the Quot functor $Quot_{3\SO}$.

Since $Hom_{X \times_k Y}(A \boxtimes B , A' \boxtimes B') \cong Hom_X(A,A') \otimes_k Hom_Y(B,B')$ we can define a morphism of sheaves $\phi:3\SO \boxtimes \mathcal{O}_{\mathbb{P}W_d} \to \SO(d) \boxtimes \mathcal{O}_{\mathbb{P}W_d}(1)$ by giving an element $\sum_{i=1}^N e_i \otimes_k f_i$ where $\{e_i\}_{i=1,...,N}$ is a basis for $W_d$ and $f_i$ its respective projective coordinates.

\begin{remark}
Let $[h] \in \mathbb{P}W_d$ be a point represented by $h = (a_1,...,a_N)$, then $\phi_{\mathbb{P}^2 \times \{[h]\} }$ corresponds to a class given by a matrix $\sum_{i=1}^N a_i e_i$, which is just $[h]$.
\end{remark}

Any $[h] \in \mathbb{P}W_d$ gives a well defined element $3\SO \twoheadrightarrow im(h)$


\subsection{Given Hilbert polynomial of Jacobian ideal of n points}
Fix an integer $d > 0$ and $P = P_{\SO(d)}-P_{\mathcal{O}_J}$ a Hilbert polynomial where $J$ is a zero dimensional scheme. Let $W_d = Hom(3\SO,\SO(d))$ as vector space and denote the (open) set of torsion-free quotients of $Quot^P(3\SO)$ by $Quot_{tf}^P(3 \SO)$.
We also define $W_{d,P} = \{h \in W_d \ | \ P(im(h)) = P \}$ where $P(im(h))$ is the Hilbert polynomial of $im(h)$, note that $W_{d,P}$ is invariant through scalar multiplication so $\mathbb{P}W_{d,P}$ is well defined.

Let $[3 \SO \overset{\phi}{\twoheadrightarrow} Q] \in Quot_{tf}^P(3\SO)$, since $Q$ is torsion-free then $rk Q = 1$, which implies that $Q \cong \mathcal{I}_J (d)$. Taking compositions yields a construction:  
\begin{equation}
    3\SO \twoheadrightarrow \mathcal{I}_J(d) \hookrightarrow \SO(d)
\end{equation}
Hence $im(h) = <F_1,F_2,F_3>$ is a twisted ideal sheaf of degree $d-1$, in fact any rank 1 sheaf with $P(\mathcal{F}) = P$ must be a twisted ideal sheaf of a $0$-th dimensional scheme with finite length $n$.
x

\begin{proposition}\label{from_quot}
Fix $P$ a Hilbert polynomial of some twisted ideal sheaf of degree $d$, then $Quot^P_{tf}(3 \SO) \hookrightarrow \mathbb{P}W_{d,P}$ set theoretically
\end{proposition}
\begin{proof}
We will show that the map $[3 \SO \overset{\phi}{\to} Q] \mapsto \beta \circ i^{**} \circ \phi$ where $\beta \circ i^{**}:Q \hookrightarrow Q^{**} \overset{\cong}{\to} \SO(d)$ is well defined up to scalar multiplication.

Indeed let $(\phi,N) \sim (\phi',N')$ from $Quot^P_{tf}(3 \SO)$ and $F = \ihom (\ihom(-,\SO),\SO)$ be the double dual functor, by definition there is an isomorphism $\epsilon: N \to N'$ such that $\epsilon \circ \phi = \phi'$, it follows then that $\epsilon^{**}:=F(\epsilon)$ is an isomorphism. Since $N$ and $N'$ are torsion-free of rank $1$, there are isomorphisms $\beta, \beta'$ to $\SO(d)$.

Apply $Hom(-,\SO(d))$ on the upper green exact sequence to get the isomorphism $Hom(\SO(d),N^{**}) \cong Hom(\SO(d),\SO(d))$, hence $\beta ' \circ \epsilon^{**}$ induces $g$ thus defining a commutative square.
\begin{center}
	\begin{tikzcd}[ampersand replacement=\&]
	    0 \arrow[rd,green] \& \& \& \& 0 \\
		\& N^{**} \arrow[dd,"\epsilon^{**}"] \arrow[r,"\beta"] \arrow[rdd,dotted,"\beta ' \circ \epsilon^{**}"] \& \SO(d) \arrow[r] \arrow[dd,blue,"g"] \& 0 \arrow[ru,green] \\
		3\SO \arrow[ru,"\phi^{**}"] \arrow[rd,"\phi'^{**}"] \& \& \& \\
		\& N^{**} \arrow[r,"\beta '"] \& \SO(d) \arrow[r] \& 0
	\end{tikzcd}
\end{center}
Finally
\begin{equation}
    g \circ \beta \circ \phi^{**} = \beta ' \circ \epsilon^{**} \circ \phi^{**} = \beta ' \circ \phi '^{**}
\end{equation}
But $g$ has to be a multiple of identity, hence $\beta \circ \phi^{**}$ and $\beta ' \circ \phi '^{**}$ are the same in $\mathbb{P}W_{d,P}$.

For injectivity suppose that $[Q,\phi],[Q',\phi '] \mapsto [h] \in \mathbb{P}W_{d,P}$, then $h = \eta \circ \phi = \eta ' \circ \phi'$ where $\eta:Q \hookrightarrow Q^{**} \overset{\cong}\to \mathcal{O}(d)$. Now, $\eta$ is injective so taking kernels shows that $ker \phi = ker \phi'$ we also have that that $im(h) = im(\phi) = im(\phi')$ thus $\eta$ induces isomorphisms $\xi:Q \to im(h)$, $\xi':Q \to im(h) $. In particular this implies that there is an isomorphim $\epsilon = {\xi'}^{-1} \circ \xi : Q \to Q'$ such that $\epsilon \circ \phi = {\xi'}^{-1} \circ \xi \circ \phi = {\xi'}^{-1} \circ h = {\xi'}^{-1} \circ \xi' \circ \phi' = \phi'$.
\end{proof}

\begin{proposition}\label{to_quot}
Fix $P$ as before, then $\mathbb{P}W_{d,P} \hookrightarrow Quot^P_{tf}(3\SO)$.
\end{proposition}
\begin{proof}
Given a projective class $[h] \in \mathbb{P}W_{d,P}$ with $h \in W_{d,P}$ let $\phi : 3\SO \to im(h)$ be the restriction to image. It induces an element of $Quot(3\SO)$ with $P(im(h)) = P$, thus $[3\SO \overset{\phi}\twoheadrightarrow im(h)] \in Quot^P(3\SO)$. This construction is injective \textcolor{red}{not sure if this is correct} since given $h,h' \in W_{d,P}$ such that $[\phi,im(h)] = [\phi',im(h')]$ are the image of each $h,h'$ implies that there is an isomorphism $\epsilon:im(h) \to im(h')$ with $\epsilon \circ \phi = \phi'$. \textcolor{red}{it seems that we can copy paste the previous proof: but $\epsilon$ induces an isomorphism $\epsilon^{**}: im(h)^{**} \to im(h')^{**}$ such that there is a commuting $\SO(d) \to \SO(d)$ which must be a multiple of identity, hence $h = \lambda h'$, same argument as before}.
\end{proof}

\textcolor{red}{obs: our morphism can be seen as sheaf:Let $\Psi:\mathbb{P}W_{d,P} \to Quot^P_{tf}(3\SO)$ be the morphism from our previous proposition, it induces a sheaf $\Tilde{\Psi}$ flat over $\mathbb{P}W_{d,P}$, im not using it now}

\begin{remark}
It is possible to show \textcolor{red}{tenho todas as contas, só nao coloquei aqui} that $\nabla:\mathbb{P}H^0 \SO(d+1) \to \mathbb{P}W_{d}$ taking $[f] \mapsto [f_i]_i$ where $f_i$ are the partial derivatives of $f$ is a morphism in an open set \textcolor{red}{acho que sao os square-free, com certeza contem eles}.

Let $\Psi:\mathbb{P}W_{d,P} \to Quot^P_{tf}(3\SO)$ be the map from proposition (\ref{to_quot}), consider the following composition:

\begin{equation}
    \mathbb{P}H^0\SO(d+1) \overset{\nabla}{\to} \mathbb{P}W_d  \overset{g}{\to} Quot_{tf}(3\SO)
\end{equation}
\textcolor{red}{it is well defined on a open set (square-free)}

where $g:[F_0,F_1,F_2] \mapsto [(F_0,F_1,F_2):3\SO \twoheadrightarrow im(F_0,F_1,F_2) = \mathcal{I}_{(F_0,F_1,F_2)}]$

\end{remark}
% gera bibliografia
\printbibliography
%fim do documento
\end{document}